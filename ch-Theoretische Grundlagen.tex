\chapter{Theoretische Grundlagen}

\section{Aerodynamische Wirkweise des Rotors}
Die aerodynamische Wirkweise des Rotors einer Horizontalachsenwindkraftanlage (HAWA) ist grundlegend für das Verständnis ihrer Funktionsweise.
Im Kern basiert die Energiegewinnung durch HAWAs auf dem Prinzip der Umwandlung von kinetischer Energie des Windes in mechanische Energie durch die Rotoren.
Diese Rotoren, im Wesentlichen große Flügel oder Blätter, sind so konstruiert, dass sie den Windstrom nutzen und dabei eine Drehbewegung erzeugen.

Das zugrundeliegende physikalische Prinzip ist das gleiche wie bei Flugzeugflügeln: der Auftrieb.
Der Rotorblattquerschnitt ist so geformt, dass eine Differenz im Luftdruck zwischen der oberen und der unteren Seite des Rotorblattes entsteht, wenn Wind darüber strömt.
Die Unterseite des Rotorblattes erfährt einen höheren Druck als die Oberseite, was zu einer Auftriebskraft führt, die das Rotorblatt nach oben zieht.
Dieser Auftrieb wird in eine Drehbewegung umgesetzt, die dann genutzt wird, um über einen Generator elektrische Energie zu erzeugen.

Ein entscheidender Aspekt bei der Auslegung von Rotoren ist die Effizienz, mit der sie die Windenergie einfangen und umwandeln.
Die Effizienz eines Windrotors wird oft durch den Leistungsbeiwert quantifiziert, der ein Maß dafür ist, welcher Anteil der im Wind enthaltenen Energie durch den Rotor extrahiert wird.
Der theoretische Höchstwert dieses Leistungsbeiwertes, bekannt als Betzscher Grenzwert, besagt, dass maximal 59,3\% der Energie des Windes durch einen perfekt designten Rotor genutzt werden können.
Diese Grenze ist fundamental für das Design und die Optimierung von Windkraftanlagen.

\section{Kennzahlen}
Der Leistungsbeiwert \( C_P \) ist eine zentrale Kenngröße in der Aerodynamik von Horizontalachsenwindkraftanlagen (HAWAs) und quantifiziert das Verhältnis der nutzbaren mechanischen Leistung \( P \) einer Windturbine zur gesamten kinetischen Leistung des Windes \( P_{\text{wind}} \) im Rotorquerschnitt. Mathematisch ausgedrückt, ist \( C_P \) definiert als:

\begin{equation}
C_P = \frac{P}{\frac{1}{2} \rho A v^3}
\end{equation}

Hierbei repräsentiert \( \rho \) die Luftdichte, \( A \) die Querschnittsfläche des Rotors und \( v \) die Windgeschwindigkeit. Die Gleichung für \( P_{\text{wind}} \) reflektiert die Tatsache, dass die Leistung des Windes proportional zum Kubus der Windgeschwindigkeit ist.

Der Betzsche Grenzwert, eine theoretische Obergrenze für \( C_P \), wird durch die Betzsche Theorie festgelegt und ist auf maximal 59,3\% beschränkt. Dieser Wert, formal ausgedrückt als \( C_{P, \text{max}} = \frac{16}{27} \), basiert auf der Annahme, dass der Wind hinter der Turbine nicht vollständig zur Ruhe kommt, sondern noch eine Restgeschwindigkeit behält.
Diese Grenze repräsentiert das Maximum der Energieextraktion unter idealen, verlustfreien Bedingungen.

In der realen Anwendung sind Windturbinen verschiedenen Verlustquellen ausgesetzt. Zu diesen gehören Profilverluste, die durch Reibungswiderstand an den Rotorblättern entstehen, und induzierte Verluste, die durch die notwendige Änderung der Windrichtung hinter der Turbine verursacht werden. Diese Verluste sind besonders relevant für die Rotorblattgestaltung, da sie die aerodynamische Effizienz des Rotors direkt beeinflussen.

Fortschrittliche Windturbinendesigns streben danach, diese Verluste durch Optimierung der Rotorblattgeometrie und -ausrichtung zu minimieren. Dazu gehört die Feinabstimmung von Parametern wie Blattwinkel, Profiltiefe und -krümmung sowie die Berücksichtigung von Betriebsbedingungen wie Windgeschwindigkeit und -richtung. Die Anwendung von Computational Fluid Dynamics (CFD) und Blade Element Momentum (BEM)-Theorien spielt eine entscheidende Rolle bei der Entwicklung von Designs, die sich dem Betzschen Grenzwert annähern und gleichzeitig reale Betriebsbedingungen berücksichtigen.

\section{Aerodynamik eines Rotorblattes}
\subsection{Grundbegriffe}
\subsection{Kräfte und Kennzahlen}
\section{Methoden der Rotorblattauslegung}
\subsection{Auslegung nach Betz und Schmitz}
\subsection{Simulation mit Blatt-Element-Methode (BEM}
\section{Besonderheiten sehr kleiner Windkraftanlagen}