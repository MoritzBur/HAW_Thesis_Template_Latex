\newenvironment{blockquote}{%
  \par%
  \leftskip=3em\rightskip=0em%
  \noindent\ignorespaces}{%
  \par\medskip}


{\LARGE\bfseries\raggedleft Abstract}
\bigskip
\normalsize


    
{\bfseries\raggedleft Moritz Burmester}

\bigskip

{\bfseries\raggedleft Keywords}
\begin{blockquote}
werferg
\end{blockquote}

{\bfseries\raggedleft Abstract}
\begin{blockquote}
In order to digitize and improve operational procedures with interacting digital twins in
aviation industry, new approaches are necessary. One approach is the use of multi-
agent systems. The objective of this thesis is to develop a concept for a digital twin using
a multi-agent system, which models a Cessna 150 as well as its pilot and the air traffic
controller. The first implementation of this system is based on the "MARS" framework.
The MARS group represents all “Multi-Agent Research and Simulation” activities at the
University of Applied Sciences Hamburg at the Department of Computer Science. The
developed and implemented concept demonstrates a first simulation approach, which
aims to model the flight operation from preflight check to takeoff. This concept and its
first implementation are presented and evaluated. Afterwards, with exploring the
potential of multi-agent systems, further recommendations for the application along the
flight operation stages are done. The result of this thesis offers a better understanding of
multi-agent system in aviation industry and acts as a first foundation for future research
with MARS in this area.
\end{blockquote}

\bigskip\bigskip

{\bfseries\raggedleft Thema der Studienarbeit}
\begin{blockquote}
werferg
\end{blockquote}

{\bfseries\raggedleft Stichworte}
\begin{blockquote}
werferg
\end{blockquote}

{\bfseries\raggedleft Kurzzusammenfassung}
\begin{blockquote}
werferg
\end{blockquote}

